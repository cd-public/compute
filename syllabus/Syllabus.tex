% Don't touch this %%%%%%%%%%%%%%%%%%%%%%%%%%%%%%%%%%%%%%%%%%%
\documentclass[11pt]{article}
\usepackage{fullpage}
\usepackage[left=1in,top=1in,right=1in,bottom=1in,headheight=3ex,headsep=3ex]{geometry}
\usepackage{graphicx}
\usepackage{float}

\newcommand{\blankline}{\quad\pagebreak[2]}
%%%%%%%%%%%%%%%%%%%%%%%%%%%%%%%%%%%%%%%%%%%%%%%%%%%%%%%%%%%%%%

% Modify Course title, instructor name, semester here %%%%%%%%

\title{CSCI 5100: Theory of Computation}
\author{Calvin Deutschbein}
\date{Spring, 2025}

%%%%%%%%%%%%%%%%%%%%%%%%%%%%%%%%%%%%%%%%%%%%%%%%%%%%%%%%%%%%%%

% Don't touch this %%%%%%%%%%%%%%%%%%%%%%%%%%%%%%%%%%%%%%%%%%%
\usepackage[sc]{mathpazo}
\linespread{1.05} % Palatino needs more leading (space between lines)
\usepackage[T1]{fontenc}
\usepackage[mmddyyyy]{datetime}% http://ctan.org/pkg/datetime
\usepackage{advdate}% http://ctan.org/pkg/advdate
\newdateformat{syldate}{\twodigit{\THEMONTH}/\twodigit{\THEDAY}}
\newsavebox{\MONDAY}\savebox{\MONDAY}{Mon}% Mon
\newcommand{\week}[1]{%
%  \cleardate{mydate}% Clear date
% \newdate{mydate}{\the\day}{\the\month}{\the\year}% Store date
  \paragraph*{\kern-2ex\quad #1, \syldate{\today} - \AdvanceDate[4]\syldate{\today}:}% Set heading  \quad #1
%  \setbox1=\hbox{\shortdayofweekname{\getdateday{mydate}}{\getdatemonth{mydate}}{\getdateyear{mydate}}}%
  \ifdim\wd1=\wd\MONDAY
    \AdvanceDate[7]
  \else
    \AdvanceDate[7]
  \fi%
}
\usepackage{setspace}
\usepackage{multicol}
%\usepackage{indentfirst}
\usepackage{fancyhdr,lastpage}
\usepackage{url}
\pagestyle{fancy}
\usepackage{hyperref}
\usepackage{lastpage}
\usepackage{amsmath}
\usepackage{layout}
\usepackage{csquotes}
\setlength\parindent{0pt}

\lhead{}
\chead{}
%%%%%%%%%%%%%%%%%%%%%%%%%%%%%%%%%%%%%%%%%%%%%%%%%%%%%%%%%%%%%%

% Modify header here %%%%%%%%%%%%%%%%%%%%%%%%%%%%%%%%%%%%%%%%%
\rhead{\footnotesize Text in header}

%%%%%%%%%%%%%%%%%%%%%%%%%%%%%%%%%%%%%%%%%%%%%%%%%%%%%%%%%%%%%%
% Don't touch this %%%%%%%%%%%%%%%%%%%%%%%%%%%%%%%%%%%%%%%%%%%
\lfoot{}
\cfoot{\small \thepage/\pageref*{LastPage}}
\rfoot{}

\usepackage{array, xcolor}
\usepackage{color,hyperref}
\definecolor{clemsonorange}{HTML}{EA6A20}
\hypersetup{colorlinks,breaklinks,linkcolor=clemsonorange,urlcolor=clemsonorange,anchorcolor=clemsonorange,citecolor=black}

\begin{document}

\maketitle

\blankline

\begin{tabular*}{.93\textwidth}{@{\extracolsep{\fill}}lr}

%%%%%%%%%%%%%%%%%%%%%%%%%%%%%%%%%%%%%%%%%%%%%%%%%%%%%%%%%%%%%%

% Modify information %%%%%%%%%%%%%%%%%%%%%%%%%%%%%%%%%%%%%%%%%
E-mail: \href{mailto:cdeutschbein@lagrange.edu}{\tt\bf cdeutschbein@lagrange.edu} & Web: \href{https://cd-public.github.io/compute/}{\tt\bf https://cd-public.github.io/compute/}  \\

Residency January 23 - 26, 2025  &  Date Range: 1/13/25 - 3/7/25 \\
\hline
\end{tabular*}

\vspace{5 mm}

% Zeroth Section %%%%%%%%%%%%%%%%%%%%%%%%%%%%%%%%%%%%%%%%%%%%

\section*{Modality and Credit Hour Compliance}

\textbf{Residency} We will meet in person at LaGrange College from 8 AM to 6 PM on 1/24-25/2025 and from 8 AM to 12 Noon on 1/26/2025. These lectures will also be recorded and posted on the course website. This is 22 of the 37.5 contact hours resulting in 3 credit hour course.

\bigskip

\textbf{Asynchronous} I will upload a weekly lecture video to YouTube, with link posted to the course website on GitHub pages, every Monday from 1/13/25 to 3/3/25. This is 15.5 of the 37.5 contact hours resulting in a 3 credit hour course.  Separately, I will be available asynchronous via email for terminating and Discord for persistent communication.

\bigskip

\textbf{Deliverables} Students will be responsible for developing a Quarto Book hosted on GitHub pages including results in LaTeX, Python and Graphviz for the theory of computation and the theory of complexity. Updates will be due every Monday at 12 midnight AOE following the residency. I will target 12 hours of effort each across weekly problem sets resulting in a 2-3 hours homework per contact hour ratio in accordance with my understanding of credit hour policy.

% First Section %%%%%%%%%%%%%%%%%%%%%%%%%%%%%%%%%%%%%%%%%%%%

\section*{Course Description}

Study of abstract models of computation, unsolvability, complexity theory, formal grammars and parsing, and other advanced topics in theoretical computer science.

% Second Section %%%%%%%%%%%%%%%%%%%%%%%%%%%%%%%%%%%%%%%%%%%

\section*{Course Materials}

\begin{itemize}
\item Course materials at \href{https://cd-public.github.io/compute/}{\tt\bf https://cd-public.github.io/compute/}
\item Optional Textbook: Sipser, Michael. Introduction to the Theory of Computation. 3rd ed. Cengage Learning, 2012. ISBN: 9781133187790
\item Supplemental Material: \href{https://ocw.mit.edu/courses/18-404j-theory-of-computation-fall-2020/pages/lecture-notes/}{\tt\bf Prof. Sipser's Lecture Notes}
\end{itemize}

% Third Section %%%%%%%%%%%%%%%%%%%%%%%%%%%%%%%%%%%%%%%%%%%

\section*{Prerequisite}
B.S. Computer Science or equivalent.

% Fourth Section %%%%%%%%%%%%%%%%%%%%%%%%%%%%%%%%%%%%%%%%%%%

\section*{Course Objectives}
\textbf{\textit{LaGrange College Student Learning Outcomes (LC SLO):}}

\begin{enumerate}
\item Students will demonstrate \underline{creativity} by approaching complex problems with innovation and from
\item Students will demonstrate \underline{critical thinking} by acquiring, interpreting, synthesizing, and evaluating
information to reason out conclusions appropriately.
\item Students will demonstrate proficiency in \underline{communication} skills that are applicable to any field of
study.
\end{enumerate}

\textbf{\textit{Student Learning Outcomes (SLO) for CISC 5100}}

\skip

\textit{All learning objectives pursuant to LC SLO \{1,2,3\} and to be assessed by homework assignments.}

\begin{enumerate}
    \item \textbf{Automata and Language Theory (12.5 contact hours)}
        \begin{itemize}
            \item \textbf{SLO 1:} Define and differentiate between finite automata (DFA, NFA, GNFA) and regular expressions, and demonstrate the equivalence between these models.
            \item \textbf{SLO 2:} Describe the capabilities and limitations of push-down automata (PDA) and context-free grammars (CFG), and apply the pumping lemma to prove that certain languages are not context-free.
        \end{itemize}
    \item \textbf{Computability Theory (12.5 contact hours)}
        \begin{itemize}
            \item \textbf{SLO 3:} Explain the concept of a Turing machine and its significance in computability theory.
            \item \textbf{SLO 4:} Discuss the Church-Turing thesis and its implications for the limits of computation.
            \item \textbf{SLO 5:} Define and distinguish between decidable and undecidable problems, and provide examples of each.
            \item \textbf{SLO 6:} Understand and apply concepts like the halting problem, reducibility, and the recursion theorem to analyze the computability of problems.
        \end{itemize}
    \item \textbf{Complexity Theory (12.5 contact hours)}
        \begin{itemize}
            \item \textbf{SLO 7:} Define and analyze the time and space complexity of algorithms using appropriate measures.
            \item \textbf{SLO 8:} Understand the significance of complexity classes P, NP, PSPACE, and NP-completeness, and discuss the implications of the P versus NP conjecture.
        \end{itemize}
\end{enumerate}

% 4.5th Section %%%%%%%%%%%%%%%%%%%%%%%%%%%%%%%%%%%%%%%%%%%

\subsection*{Assignments and Assessment}

\begin{itemize}
    \item Assignments in this course will have the sole purpose of student learning. Students will:
        \begin{itemize}
            \item  Begin the course with a grade of an "A".
            \begin{itemize}
                \item  Be expected to attend class.
                \item  Be expected to participate in class.
                \item  Be expected to complete assignments.
            \end{itemize}
            \item  Be contacted privately by the course instructor in the unusual event they are not meeting expectations.
            \begin{itemize}
                \item  Not have an expectation of perfection.
                \item  Not lose points or a grade without discussion.
                \item  Have a chance to explain their engagement with the course.
            \end{itemize}
            \item  Receive feedback collectively from the instructor and individually from peers.
            \begin{itemize}
                \item  Be entitled to individual feedback from the instructor at any time.
                \item  Receive narrative rather than quantitative feedback.
                \item  Receive positive and constructive feedback only.
            \end{itemize}
        \end{itemize}
\end{itemize}

% Fifth Section %%%%%%%%%%%%%%%%%%%%%%%%%%%%%%%%%%%%%%%%%%%

\section*{College Policies}

\subsection*{ADA Statement}

In compliance with Section 504 of the Rehabilitation Act and the Americans with Disabilities Act,
LaGrange College will provide reasonable accommodation of all medically documented disabilities. If you
have a disability and would like the College to provide reasonable accommodations of the disability during
this course, please notify Ms. Lindsay Shaughnessy, Director of the Panther Academic Center for
Excellence (PACE) and Coordinator of Accessibility Services at \href{mailto:accessability@lagrange.edu}{accessability@lagrange.edu} or 706-880-8652. PACE is located in the Moshell Learning Center &amp; Tutoring Lab in the Lewis Library.


\subsection*{Academic Support}

Academic Support
Academic support at LaGrange is provided through Panther Academic Center for Excellence (PACE), the
Writing Center, and the advising deans. PACE provides peer tutoring, testing services, accessibility
services, and other academic support as needed. For more information about PACE, please contact Mr.
Steve Kenner (skenner@lagrange.edu). The Writing Center gives all writers a space to explore the potential of their ideas via peer review. For information about the Writing Center, contact Dr. Justin Thurman
(jthurman@lagrange.edu).

\subsection*{Academic Integrity Policy}

Each student is bound by the LaGrange College Honor Code which is stated as follows:

\begin{displayquote}
As a member of the student body of LaGrange College, I confirm my
commitment to the ideals of civility, diversity, service, and excellence.
Recognizing the significance of personal integrity in establishing these ideals
within our community, I pledge that I will not lie, cheat, steal, nor tolerate these
unethical behaviors in others.
\end{displayquote}

The full text of the LaGrange College Honor Code along with policies and procedures in cases of academic
dishonesty can be found at \url{http://www.lagrange.edu/resources/pdf/honorcode12-13.pdf}.

\subsection*{Academic Integrity Policy}

Email and LaGrange college accounts will be used in accordance with the following student handbook
statement:

\begin{displayquote}
“Students are expected to treat their campus [e­mail] accounts as a business account.
Faculty and administrators rely on these accounts to disseminate important information regarding
College protocol and events; therefore, students are responsible for any College information sent
out over campus e­mail.”
\end{displayquote}

Consequently, personal email addresses will not be used for instructor/student email contact except in event a service interruption. The preferred
method of contact will instead be by the official campus email. I target a 24 hour maximum response time on school days and 48 hours maximum response time on all emails while the course is active.

\skip

As an adjunct, my LaGrange email may not persistent indefinitely. I maintain a persistent professional email at \href{calvindeu@gmail.com}{mailto:calvindeu@gmail.com} which can also be used in event service interrupts to the campus network or for professional references after the conclusion of the course.

\subsection*{Netiquette}

When leaving comments or asking questions in the forums of an online course, one is reminded to observe
a few rules of internet etiquette:

\begin{itemize}
\item All caps locks and/or multiple exclamation points typically imply anger. You should not use such
emphases unless it accomplishes a learning objective.
\item Vulgarity, rudeness, and/or disrespect are complete unacceptable and will not be tolerated.
\item Emoticons (such as ‘:)’ for a ‘smiley face’) are fine for use in relaxed submissions (forum threads
and posts).
\item In general, do your best to use proper spelling, grammar, and punctuation. Writing correctly
works to ensure that your meaning is conveyed.
\end{itemize}

\subsubsection*{Technology Requirements}

It is technically possible to complete the assessed work on this course working fully within a web-browser through a combination of cloud services and other technologies, the methodologies for which are left to the interested student as an exercise. That said, I recommend each student utilize the following technology stack:

\begin{itemize}
\item A local installation of Python, at least 3.8, ideally more recent.
\item A local installation of VS Code.
\item A local installation of Quarto.
\item A local installation of a version control client compatible with GitHub.
\item A local installation of major desktop webbrowser, likely Firefox, Chrome, Safari, or Edge.
\item A remote GitHub account affilitated with an official LaGrange.edu email address.
\end{itemize}

\subsubsection*{Technical Support}

I will independently offer technology support for the technology stack used to support this course. Contact me directly unless you have technical issues arising within LaGrange.edu realms, in which case you should reach out via email to \href{support@lagrange.edu}{mailto:support@lagrange.edu} or call 706.880.8053.

\skip

Precise technical writing is a core learning objective (LC SLO 3) for this course, and should be modeled in all technical support interactions.

\subsubsection*{Agreement by Continued Enrollment}

By remaining enrolled in the course, each student agrees to the terms of the syllabus as a binding contract between the student, the instructor, and LaGrange College.

\subsubsection*{Note on attending asynchronous attendance:}

I am confident I have formulated the assessment tools such that attendance or non-attendance by individuals, as measured by viewing of asynchronous lectures, will be obvious to me as an instructor. As such, I have folded my attendance considerations into the assessment formulation.

\skip

It is trivial as an instructor to assess the level of engagement with asynchronous learning resources, and you should regard it as more, not less, clear what a student's level of participation is for asynchronous instruction.

\end{document}
