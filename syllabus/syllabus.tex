% Don't touch this %%%%%%%%%%%%%%%%%%%%%%%%%%%%%%%%%%%%%%%%%%%
\documentclass[11pt]{article}
\usepackage{fullpage}
\usepackage[left=1in,top=1in,right=1in,bottom=1in,headheight=3ex,headsep=3ex]{geometry}
\usepackage{graphicx}
\usepackage{float}
\usepackage{quoting}

\newcommand{\blankline}{\quad\pagebreak[2]}
%%%%%%%%%%%%%%%%%%%%%%%%%%%%%%%%%%%%%%%%%%%%%%%%%%%%%%%%%%%%%%

% Modify Course title, instructor name, semester here %%%%%%%%

\title{DATA 597: Automata}
\author{Calvin Deutschbein}
\date{Fall, 2025}

%%%%%%%%%%%%%%%%%%%%%%%%%%%%%%%%%%%%%%%%%%%%%%%%%%%%%%%%%%%%%%

% Don't touch this %%%%%%%%%%%%%%%%%%%%%%%%%%%%%%%%%%%%%%%%%%%
\usepackage[sc]{mathpazo}
\linespread{1.05} % Palatino needs more leading (space between lines)
\usepackage[T1]{fontenc}
\usepackage[mmddyyyy]{datetime}% http://ctan.org/pkg/datetime
\usepackage{advdate}% http://ctan.org/pkg/advdate
\newdateformat{syldate}{\twodigit{\THEMONTH}/\twodigit{\THEDAY}}
\newsavebox{\MONDAY}\savebox{\MONDAY}{Mon}% Mon
\newcommand{\week}[1]{%
%  \cleardate{mydate}% Clear date
% \newdate{mydate}{\the\day}{\the\month}{\the\year}% Store date
  \paragraph*{\kern-2ex\quad #1, \syldate{\today} - \AdvanceDate[4]\syldate{\today}:}% Set heading  \quad #1
%  \setbox1=\hbox{\shortdayofweekname{\getdateday{mydate}}{\getdatemonth{mydate}}{\getdateyear{mydate}}}%
  \ifdim\wd1=\wd\MONDAY
    \AdvanceDate[7]
  \else
    \AdvanceDate[7]
  \fi%
}
\usepackage{setspace}
\usepackage{multicol}
%\usepackage{indentfirst}
\usepackage{fancyhdr,lastpage}
\usepackage{url}
\pagestyle{fancy}
\usepackage{hyperref}
\usepackage{lastpage}
\usepackage{amsmath}
\usepackage{layout}
\usepackage{csquotes}
\setlength\parindent{0pt}

\lhead{}
\chead{}
%%%%%%%%%%%%%%%%%%%%%%%%%%%%%%%%%%%%%%%%%%%%%%%%%%%%%%%%%%%%%%

% Modify header here %%%%%%%%%%%%%%%%%%%%%%%%%%%%%%%%%%%%%%%%%
\rhead{\footnotesize Text in header}

%%%%%%%%%%%%%%%%%%%%%%%%%%%%%%%%%%%%%%%%%%%%%%%%%%%%%%%%%%%%%%
% Don't touch this %%%%%%%%%%%%%%%%%%%%%%%%%%%%%%%%%%%%%%%%%%%
\lfoot{}
\cfoot{\small \thepage/\pageref*{LastPage}}
\rfoot{}

\usepackage{array, xcolor}
\usepackage{color,hyperref}
\definecolor{clemsonorange}{HTML}{EA6A20}
\hypersetup{colorlinks,breaklinks,linkcolor=clemsonorange,urlcolor=clemsonorange,anchorcolor=clemsonorange,citecolor=black}

\begin{document}

\maketitle

\blankline

\begin{tabular*}{.93\textwidth}{@{\extracolsep{\fill}}lr}

%%%%%%%%%%%%%%%%%%%%%%%%%%%%%%%%%%%%%%%%%%%%%%%%%%%%%%%%%%%%%%

% Modify information %%%%%%%%%%%%%%%%%%%%%%%%%%%%%%%%%%%%%%%%%
E-mail: \href{mailto:ckdeutschbein@willamette.edu}{\tt\bf ckdeutschbein@willamette.edu} & Web: \href{https://cd-public.github.io/compute/}{\tt\bf https://cd-public.github.io/compute/}  \\
\hline
\end{tabular*}

\vspace{5 mm}

% Zeroth Section %%%%%%%%%%%%%%%%%%%%%%%%%%%%%%%%%%%%%%%%%%%%

\section*{Modality and Credit Hour Compliance}

\bigskip

\textbf{Student support} As a course intended to be completed with in person lecture with student questions, course TAs, and study groups, I will be available to meet weekly with students, 1-on-1, to go over course material and homeworks.

\bigskip

\textbf{Asynchronous} I will upload a weekly lecture video to YouTube, with link posted to the course website on GitHub pages, every Monday from 8/25/25 to 11/17/25. Each lecture will cover the equivalent 4 credit hours of content, and is intended to be paused or replayed when students would typically ask questions in class. Together, this is 56 contact hours.

\bigskip

\textbf{Deliverables} Students will be responsible for developing a Quarto Book hosted on GitHub pages expressing their own presentation of major results in automata theory using, LaTeX, Python and Graphviz. Updates will be due on Monday at 12 midnight AOE. I will target 6-12 hours of effort each across week in accordance with my understanding of credit hour policy. Besides the Quarto Book, there will be regular exercises on weeks supporting intermediate results to scaffold the the major results.


% First Section %%%%%%%%%%%%%%%%%%%%%%%%%%%%%%%%%%%%%%%%%%%%

\section*{Course Description}

Study of abstract models of computation, unsolvability, complexity theory, formal grammars and parsing, and other advanced topics in theoretical computer science.

% Second Section %%%%%%%%%%%%%%%%%%%%%%%%%%%%%%%%%%%%%%%%%%%

\section*{Course Materials}

\begin{itemize}
\item Course materials at \href{https://cd-public.github.io/compute/}{\tt\bf https://cd-public.github.io/compute/}
\item Optional Textbook: Sipser, Michael. Introduction to the Theory of Computation. 3rd ed. Cengage Learning, 2012. ISBN: 9781133187790
\item Supplemental Material: \href{https://ocw.mit.edu/courses/18-404j-theory-of-computation-fall-2020/pages/lecture-notes/}{\tt\bf Prof. Sipser's Lecture Notes}
\end{itemize}

% Third Section %%%%%%%%%%%%%%%%%%%%%%%%%%%%%%%%%%%%%%%%%%%

\section*{Prerequisite}
B.S. Computer Science or equivalent.

% Fourth Section %%%%%%%%%%%%%%%%%%%%%%%%%%%%%%%%%%%%%%%%%%%

\section*{Course Objectives}

\begin{enumerate}
    \item \textbf{Automata and Language Theory (18 contact hours)}
        \begin{itemize}
            \item \textbf{SLO 1:} Define and differentiate between finite automata (DFA, NFA, GNFA) and regular expressions, and demonstrate the equivalence between these models.
            \item \textbf{SLO 2:} Describe the capabilities and limitations of push-down automata (PDA) and context-free grammars (CFG), and apply the pumping lemma to prove that certain languages are not context-free.
        \end{itemize}
    \item \textbf{Computability Theory (18 contact hours)}
        \begin{itemize}
            \item \textbf{SLO 3:} Explain the concept of a Turing machine and its significance in computability theory.
            \item \textbf{SLO 4:} Discuss the Church-Turing thesis and its implications for the limits of computation.
            \item \textbf{SLO 5:} Define and distinguish between decidable and undecidable problems, and provide examples of each.
            \item \textbf{SLO 6:} Understand and apply concepts like the halting problem, reducibility, and the recursion theorem to analyze the computability of problems.
        \end{itemize}
    \item \textbf{Complexity Theory (18 contact hours)}
        \begin{itemize}
            \item \textbf{SLO 7:} Define and analyze the time and space complexity of algorithms using appropriate measures.
            \item \textbf{SLO 8:} Understand the significance of complexity classes P, NP, PSPACE, and NP-completeness, and discuss the implications of the P versus NP conjecture.
        \end{itemize}
\end{enumerate}

% 4.5th Section %%%%%%%%%%%%%%%%%%%%%%%%%%%%%%%%%%%%%%%%%%%

\subsection*{Assignments and Assessment}

\begin{itemize}
    \item Assignments in this course will have the sole purpose of student learning. Students will:
        \begin{itemize}
            \item  Begin the course with a grade of an "A".
            \begin{itemize}
                \item  Be expected to study course materials.
                \item  Be expected to proactively reach out to the professor with any questions..
                \item  Be expected to complete assignments.
            \end{itemize}
            \item  Be contacted privately by the course instructor in the unusual event they are not meeting expectations.
            \begin{itemize}
                \item  Not have an expectation of perfection.
                \item  Not lose points or a grade without discussion.
                \item  Have a chance to explain their engagement with the course.
            \end{itemize}
            \item  Receive feedback collectively from the instructor and individually from peers.
            \begin{itemize}
                \item  Be entitled to individual feedback from the instructor at any time.
                \item  Receive narrative rather than quantitative feedback.
                \item  Receive positive and constructive feedback only.
            \end{itemize}
        \end{itemize}
\end{itemize}

\subsection*{College Policies}

The following material is adapted from ``Information for Syllabus'' recommended language on syllabus prepartion provided to insturctors in the College of Arts \& Sciences.

\subsubsection*{Academic Integrity}

Students of Willamette University are members of a community that values excellence and integrity in every aspect of life. As such, we expect all community members to live up to the highest standards of personal, ethical, and moral conduct. Students are expected not to engage in any type of academic or intellectually dishonest practice and encouraged to display honesty, trust, fairness, respect, and responsibility in all they do. Plagiarism and cheating are especially offensive to the integrity of courses in which they occur and against the College community as a whole. These acts involve intellectual dishonesty, deception, and fraud, which inhibit the honest exchange of ideas. Plagiarism and cheating may be grounds for failure in the course and/or dismissal from the College. \url{http://willamette.edu/cla/catalog/policies/plagiarism-cheating.php}

\subsubsection*{Commitment to Positive Sexual Ethics}

Willamette is a community committed to fostering safe, productive learning environments, and we value ethical sexual behaviors and standards. Title IX and our school policy prohibit discrimination on the basis of sex, which regards sexual misconduct — including discrimination, harassment, domestic and dating violence, sexual assault, and stalking. We understand that sexual violence can undermine students’ academic success, and we encourage affected students to talk to someone about their experiences and get the support they need. 

\begin{quoting}\textbf{Please be aware that as a mandatory reporter I am required to report any instances you disclose to Willamette's Title IX Coordinator.}\end{quoting}

If you would rather share information with a confidential employee who does not have this responsibility, please contact our confidential advocate at confidential-advocate@willamette.edu. Confidential support also can be found with SARAs and at the GRAC (503-851-4245); and at WUTalk - a 24-hour telephone crisis counseling support line (503-375-5353). If you are in immediate danger, you may reach campus safety at 503-370-6911.

\subsubsection*{DACA/Undocumented Student Advocate}

Willamette is committed to supporting our DACA/Undocumented students in a variety of ways. This year, Tori Ruiz is the contact person for all DACA/undocumented students can provide those students with a number of external and internal resources that are available. Her contact information is email:~\href{mailto:truiz@willamette.edu}{truiz@willamette.edu}, Office: 3rd Floor UC, Phone: 503-370-6447.

\subsubsection*{Diversity and Disability Statement}

Willamette University values diversity and inclusion; we are committed to a climate of mutual respect and full participation. My goal is to create a learning environment that is usable, equitable, inclusive and welcoming. If there are aspects of the instruction or design of this course that result in barriers to your inclusion or accurate assessment or achievement, please notify me as soon as possible. Students with disabilities are also encouraged to contact the Accessible Education Services office in Smullin 155 at 503-370-6737 or Accessible-info@willamette.edu to discuss a range of options to removing barriers in the course, including accommodations.

\subsubsection*{Religious Practice}

Willamette University recognizes the value of religious practice and strives to accommodate students’ commitment to their religious traditions whenever possible. Please let me know within the first two weeks of the semester if a conflict between holy days or other religious practice and full participation in the course is anticipated. I will do my best to work with you to determine a reasonable accommodation.

\textit{As an instructor, I will exercise my discretion to offer accomodations for conflicts after the first two weeks of the semester. You may always reach out to me, including retroactively, though the quality of the accomodation I am able to offer may improve given advanced warning!}

\subsubsection*{SOAR Center Offerings: Food, Clothing, and School Materials}

The Students Organizing for Access to Resources (SOAR) Center strives to create equitable access to food, professional clothing, commencement regalia, and scholarly resources for WU and Willamette Academy students. The SOAR Center is located on the Putnam University Center's third floor (in the former Women's Resource Center and across from the Harrison Conference Room). The space houses the Bearcat Pantry, Clothing Share, and First-Generation Book Drive and is maintained by committed students and staff and faculty advisers.

\subsubsection*{Trans Inclusion and Gender Justice}

I am always appreciative of the opportunity to address you by your affirming name or pronoun. Please advise me of the most affirming way to address you at any time so that I may do so.

If I ever misgender you in any way, I would greatly appreciate that you let me know, in whatever manner makes you comfortable, so that I can correct that error and endeavour to repair any harm. 

\subsubsection*{Mental Health}
As a student, you may experience a range of challenges that can interfere with learning, such as strained
relationships, increased anxiety, substance use, feeling down, difficulty concentrating and/or lack of
motivation. These mental health concerns or stressful events may diminish your academic performance and/or
reduce your ability to participate in daily activities. Willamette services are available and treatment does work.
If you think you need help, please contact Bishop Health as soon as possible at
\url{http://willamette.edu/offices/counseling/}. Crisis counseling is available 24/7 at WUTalk: 503-375-5353 and
Campus Safety is available at 503-370-6911. Emergency resources are also available from the Psychiatric
Crisis Center at 503-585-4949 and the National Suicide Prevention Lifeline at 1-800-273-8255.


\end{document}
